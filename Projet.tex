\documentclass[frenchb]{article}
	
	\usepackage[utf8]{inputenc}
	%\usepackage{mathaccent}
	\usepackage{xspace}
	\usepackage[francais]{babel}
	
	\begin{document}
		\begin{title}
			\textbf{\huge{Projet Flot et combinatoire}}
		\end{title}
		
		
		\section {Enoncé}
			Lors de la conception de réseaux de télécommunications, il est nécessaire non seulement de mettre en oeuvre des liaisons entre toutes les paires possibles de terminaux, mais également d'assurer le fonctionnement du réseau lors de la panne d'un lien. A cette fin, on va imposer, pour chacun des liens, qu'il existe un autre chemin, de longueur limitée, permettant de rerouter l'information en cas de panne du lien. Connaissant le coût d'établissement des liens, cette topologie de réseau doit être conçue pour un coût minimum.
			\begin{itemize}
				\item{Modéliser ce problème dans un graphe}
				\item{Quelle structure permet d'avoir toutes les communications possibles pour un coût minimum ?}
				\item{Comment mettre en place la fiabilité du réseau en cas de panne ?}
			\end{itemize}
			
			
			\section{Analyse}
				Pb de conception de réseaux:\\
				Ens de terminaux. Pb est de relier ces terminaux entre eux pour un coût minimum de manière à ce que:
				\begin{itemize}
					\item{Il existe un chemin entre toutes paures de terminaux}
					\item{Ce réseau soit résistant aux pannes (Si un certain nombre d'arêtes tombent en panne, il faut que le réseau reste connecté)} 
				\end{itemize}
				
				Une condition qui s'est révélée suffisante est que le réseau soit 2 arêtes connexes (entre toute paire de sommets, il existe 2 chemins arête-disjoints)\\
				
				On va imposer que chaque lien du réseau appartienne à un cycle de longueur bornée par un nombre K\\
				Données du problème:\\
				
				Graphe = (V,E) \\
				V: ens des terminaux.\\
				E: ens des liens possibles.\\
				(on suppose que le graphe est complet (donc $E=V*V$) On peut créer tous les liens que l'on veut)\\
				
				$C_{ij}$ = Cout de construction du lien i,j $\forall $ i,j $\in E $\\
				
				$\rightarrow $K, un entier (borne sur les cycles)\\
				
				Pb: déterminer les liens de E à créer tel que le graphe soit connexe et que chaque arête appartienne à un cycle de longueur $\le K$ tout en minimisant le coût.\\
				
				
				
				Données: \# noeud coord\_x  coord\_y\\
				
				$C_{ij}$ : distance euclidienne entre les extrémités.\\
				
				ulbnodes $\rightarrow$ instance réelle de belgacom.\\
				
				
				$
				\left.
				\begin{array}{l}
					\mbox{nodes1}\\
					\mbox{nodes2}\\
					\mbox{nodes3}
				\end{array}
				\right \}$sous ensemble de ulbnodes 
				
				
				n20-3 $\rightarrow$ instances aleatoires (20 noeuds , inst \#3)
				
				
				
				Problème de couverture de l'exo 2 du td 2.\\
				Rapport: 
				\begin{itemize}
					\item Introduction présentant le problème.\\
					\item Explication de /des heuristiques que l'on propose.(algos , explication de l'algo et des éléments de base utilisées par l'algo)\\
					\item Résultats numériques+analyse des résultats numériques.
					\item conclusion
				\end{itemize}
				Rapport + Code par mail pour le \textbf{17 Mai}
			\section{Première heuristique}
                            Afin d'appliquer un algorithme génétique, il a fallu générer une génération 0. Pour cela nous avons décidé d'implémenter une heuristique simple, qui permet d'obtenir une solution naïve mais valide. L'algorithme utilisé est le suivant : \begin{itemize}
                        \item Calculer un arbre couvrant minimal.
                        \item Organiser l'ensemble des noeuds en différents niveaux en partant d'un noeud donné.
                        \item Partir du plus bas niveau (contenant les feuilles les plus profondes) et créer des cycles de longueur K en remontant dans l'arbre sur K-1 niveaux
                        \item Recommencer pour les noeuds du niveau supérieur
                        \item Pour l'avant dernier niveau, faire des liens entre les branches de longueur 1
                        \end{itemize}
                        Cela permet d'obtenir un graphe solution, puisque tous les noeuds ont été incorporés dans des cycles de longueur K ou moins.
                        \section{Algorithmes génétiques}
                        Les algorithmes génétiques sont souvent utilisés pour des problèmes d'optimisation où il n'est pas possible de trouver la solution exacte en temps polynomiale, comme par exemple le Voyageur de Commerce. \\
                        \subsection{Principe des algorithmes génétiques}
                        On part d'une population de base. Chaque individu est composé d'un chromosome (ou plus). Un individu correspond à une solution réalisable du problème. Chaque chromosome est composé de gênes qui sont des unités atomiques de la solution.
                        A partir de cette population de base, qui est la génération 0, on va créer une nouvelle génération à l'aide de deux opérations : crossover et mutation.
                        \begin{itemize}
                         \item Crossover : Il s'agit de créer un individu à partir de deux individus de la génération précédente. On peut par exemple prendre pour chaque gêne du chromosome une version au hasard du premier parent ou du second. Il s'agit de créer une solution réalisable :
                         \item Mutation : Il s'agit de créer un individu à partir d'un individu de la génération précédente en lui appliquant des mutations, c'est à dire des changements aléatoires. Il faut que la solution soit toujours valide. \\\end{itemize}
                   
                        Pour que ces opérations soient possibles il faut que chaque chromosome soit "codé" d'une façon simple, le plus simple est de prendre un codage binaire d'une taille définie (nombre de noeuds par exemple).
                        Avec ces nouveaux individus, il faut sélectionner la nouvelle génération. On définit une fonction "fitness" qui permet de mesurer la valeur d'une solution. Une solution simple est de garder les N meilleurs, avec N entier naturel.


                       \subsection {Dans notre cas}
                        Nous avons d'abord réfléchit à un codage binaire simple : pour chaque arrête i, le ième bit du chromosome est à 1 si i appartient à la solution, 0 sinon. Toutefois le problème est que le croisement de cette solution donne difficilement des solutions réalisables. 
                        Notre codage est plus compliqué : Pour chaque noeud de la liste d'un noeud. Par exemple pour le noeud 4 [4,3,23,5] pour un cycle de longueur 4.
                        On remarque qu'un noeud peut avoir plusieurs cycles. On pourrait retravailler à chaque fois nos solutions pour avoir tous les cycles. Sur un graphe où les noeuds ont peu de voisins, c'est une solution.
                        \section{Analyse du résultat}
                        Un problème qu'on a eu est de démarrer l'algorithme génétique, car il faut des suffisament différentes pour créer la première nouvelle génération.
			\section{Conclusion}
				
				
			\end{document}			
